\documentclass[a5paper]{article}
\usepackage[cp1251]{inputenc}
\usepackage[english]{babel}
\usepackage{amssymb, amsmath, amsthm, eucal, latexsym}
\usepackage{indentfirst}
\usepackage{enumerate}

\addto{\captionsrussian}{\renewcommand{\refname}{}}
\addto{\captionsenglish}{\renewcommand{\refname}{}}

\def\newarticle#1#2#3{
\vskip 0.1cm
\begin{center}
{\bf #1}\\
\vskip 0.1cm
{\bf #2}\\ #3
\end{center}
\addcontentsline{toc}{section}{{\it #2} #1}
\vskip 0.1cm
\setcounter{equation}{0}
\setcounter{figure}{0}
\setcounter{footnote}{0}}
\newcounter{foref}[section]

\usepackage[top=1.6cm, bottom=1.8cm, left=1.6cm, right=1.6cm]{geometry}

\setcounter{page}{1}

\newtheorem*{theorem}{Theorem}

\begin{document}
\selectlanguage{english}
\newarticle{Coding of bounded solutions of equation $u_{xx} - u + \eta(x) u^3 = 0$ with periodic piecewise constant function $\eta(x)$}{M. E. Lebedev, G. L. Alfimov}{MIET University, Zelenograd, Moscow, Russia}

Consider a one-dimensional second-order differential equation
\begin{equation}
	u_{xx} - u + \eta(x) u^3 = 0,
\label{EqMain}
\end{equation}
where $\eta(x)$ is a periodic piecewise-constant function of period $L + \ell$,
\begin{equation}
	\eta(x)=\left\{
		\begin{array}{rl}
			-1, &  x \in [0; L]; \\[1.5mm]
			\xi, & x \in [L; L + \ell],
		\end{array}
	\right.
\label{W(x)}
\end{equation}
where $\xi > 0$.
Let us define two topological spaces.

\underline{At first}, denote by $\mathcal{S}(b), \, b \in \mathbb{R}$, a set of solutions for equation (\ref{EqMain}) such that $|u(x)| < b$ on the whole real axis $\mathbb{R}$.
Evidently, $b_1 < b_2$ implies $\mathcal{S}(b_1) \subseteq \mathcal{S}(b_2)$.
One can define a metric $\rho$ in $\mathcal{S}(b)$ as follows,
\begin{equation}
	\rho(v, w)=\sqrt{(v(0) - w(0))^2 + (v_x(0) - w_x(0))^2}, \quad v(x), \, w(x) \in \mathcal{S}(b). 
\end{equation}
This implies that $\mathcal{S}(b)$ can be regarded as topological space where neighbourhood $U_\varepsilon(u)$ of an element $u \in \mathcal{S}(b)$ is defined as $U_\varepsilon(u) = \{ v|\ \rho(u, v) < \varepsilon \}$.

\underline{At second}, denote by $\Omega_n$ the set of bi-infinite sequences $\{\ldots,i_{-1},i_0,i_1,\ldots\}$ where $i_k$, $k=0,\pm1,\ldots$, is an integer, $-n\leq i_k\leq n$.
Evidently that for $n_1 < n_2$ one has $\Omega_{n_1} \subset \Omega_{n_2}$.
The set $\Omega_n$ can be regarded as topological space where neighbourhood $W_k (\omega^*)$ of an element $\omega^* = \{ \dots, i_{-1}^*, i_0^*, i_1^*, \dots \} \in \Omega_n$ is defined as $W_k(\omega^*) = \{ \omega |\ i_s^* = i_s, \, |s| < k \}$.

The main result of our study is the following theorem.
\begin{theorem}
For any $N$ there exists a pair $(L_0, \ell_0)$ such that for any pair $(L, \ell)$, $L > L_0$ and $0 < \ell < \ell_0$, there exist a sequence
\begin{equation*}
	b_0 < b_1 < \ldots < b_N,
\end{equation*}
and a homeomorphism $T$ such that $T \mathcal{S}(b_n) = \Omega_n$, $n = 0, 1, \ldots, N$.
\end{theorem}

The theorem can be illustrated by the following diagram:
\begin{equation}
\begin{array}{lllllll}
	\mathcal{S}(b_0) & \subset & \mathcal{S}(b_1) & \subset & \cdots & \subset & \mathcal{S}(b_N) \\ [2mm]
	\downarrow{T} & & \downarrow{T} & & & & \downarrow{T} \\ [2mm]
	\Omega_0 & \subset & \Omega_1 & \subset & \cdots & \subset & \Omega_N
\end{array}
\end{equation}

The theorem is proved for $\xi$ below a threshold $\xi_0$ that is a root of some transcendent equation.
\end{document} 