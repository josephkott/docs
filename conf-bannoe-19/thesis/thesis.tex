\documentclass[a5paper]{article}
\usepackage[T1,T2A]{fontenc}
\usepackage[utf8]{inputenc}
\usepackage[english,russian]{babel}
\usepackage{amssymb,amsmath,eucal,latexsym,amsthm}
\usepackage{indentfirst}
\usepackage{enumerate}

%\usepackage{ifmtarg}
\addto{\captionsrussian}{\renewcommand{\refname}{}}
\addto{\captionsenglish}{\renewcommand{\refname}{}}

\def\newarticle#1#2#3{
\vskip 0.1cm
\begin{center}
{\bf #1}\\
%\if{#4=''}\\\else\footnote{#4}\\\fi
\vskip 0.1cm
{\bf #2}\\ #3
\end{center}
\addcontentsline{toc}{section}{{\it #2} #1}
\vskip 0.1cm
\setcounter{equation}{0}
\setcounter{figure}{0}
\setcounter{footnote}{0}}
\newcounter{foref}[section]

\usepackage[top=1.6cm, bottom=1.8cm, left=1.6cm, right=1.6cm]{geometry}

\setcounter{page}{1}


\begin{document}

\newarticle
{Symmetry breaking in competing single-well linear-nonlinear potentials}
{
	D.~A.~Zezyulin\textsuperscript{a},
	M.~E.~Lebedev\textsuperscript{b,c},
	G.~L.~Alfimov\textsuperscript{c,d},
	and~B.~A.~Malomed\textsuperscript{a,e}
}
{
	\textsuperscript{a} ITMO University, St. Petersburg, Russia; \\
	\textsuperscript{b} All-Russian Institute for Scientific and Technical Information, RAS, Moscow, Russia; \\
	\textsuperscript{c} Institute of Mathematics with Computer Center, RAS, Ufa, Russia; \\
	\textsuperscript{d} MIEE University, Zelenograd, Moscow, Russia; \\
	\textsuperscript{e} Tel Aviv University, Israel
}

The combination of linear and nonlinear potentials, both shaped as a single well, enables competition between the confinement and expulsion induced by the former and latter potentials, respectively.
We demonstrate that this setting leads to spontaneous symmetry breaking (SSB) of the ground state in the respective generalized nonlinear Schr\"odinger (Gross-Pitaevskii) equation:

\begin{equation}
i \Psi_t = -\Psi_{xx} + \frac{1}{2} \omega^2 x^2 \Psi - P(x) \Psi |\Psi|^2.
\end{equation}

Two different SSB bifurcation scenarios are possible, depending on the shape of the nonlinearity -- modulation profile, which determines the nonlinear potential $P(x)$.
If the profile is bounded (remaining finite at $|x| \to \infty$), at a critical value of the integral norm the spatially symmetric state loses its stability, giving rise to a pair of mutually symmetric stable asymmetric ones via a direct pitchfork bifurcation.
On the other hand, if the nonlinear potential is unbounded, two unstable asymmetric modes merge into the symmetric metastable mode and destabilize it via an inverted pitchfork bifurcation.

\vspace{-1cm}

\begin{thebibliography}{99}
\bibitem{ZLAM2018} D. A. Zezyulin, M. E. Lebedev, G. L. Alfimov, and B. A. Malomed, Symmetry breaking in competing single-well linear-nonlinear potentials, Phys. Rev. E \textbf{98}, 042209 (2018).
\end{thebibliography}

\end{document}
